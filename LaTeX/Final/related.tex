\section{Related Work}
\label{sec:related}

The idea of using machine learning to colorize images first became popular in the early 2000s.  Welsh et al. began with an algorithm which transferred colors simply based on the value and standard deviation of luminance at each pixel \cite{welsh2002transferring}.  Although the algorithm runs quickly, the results we obtained by implementing their method are far from optimal in that each prediction is purely local and that the feature space they are searching over has only two dimensions.  Soon after, Levin et al. took colorization in a completely different direction with their reformulation as a global optimization problem over the image \cite{levin2004colorization}.  Unlike Welsh's technique, which accepted a color image as training input, Levin's optimization algorithm accepted hand-drawn ``scribbles" of color from the user which are then used as linear constraints for the optimization of a quadratic cost function.  This algorithm sought to label pixels the same color if they had similar luminance values, which is one of the first examples of automatic colorization algorithms explicitly incorporating spatial coherency as part of their classification.

Irony et al. then improved the technique by incorporating a two-phase voting procedure \cite{irony2005colorization}.  In the first step, pixels are independently colored based on their individual texture features.  In the second, each pixel's color is replaced by a weighted majority vote over its neighboring pixels.  The most confidently labelled pixels are then fed into Levin's original algorithm as ``micro-scribbles."  Noda et al. formulated the problem as Bayesian inference using a Markov random field model of an image \cite{noda2005bayesian}.  This formulation naturally includes spacial coherency considerations while computing the maximum a posteriori (MAP) estimate of the colors.  Finally, one of the more recent takes on the colorization problem was presented by Charpiat et al., on whose algorithm ours is based \cite{charpiat2010machine}.  This technique involves solving an energy minimization problem, the details of which will be laid out in Section \ref{sec:methods}.